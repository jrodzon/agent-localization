
\chapter{Preface}

We live in the era of automation.
Everywhere robots are taking over the positions, which require not only repetitive actions but also more complicated tasks.
In this environment, there are more and more machines that are operating within some contained spaces like commercial buildings or warehouses.
There is also a growing market for indoor drone usage.
Because of that, there is a need for a means of autonomous localization of these machines, which we will call generically agents in this work.
Recent advances in computer vision and artificial intelligence come in handy and provide some models for localization based on planar segments.

\section{Motivation}

Unfortunately, when one tries to use or apply the latest state-of-the-art models and algorithms there are a lot of obstacles to it—starting from
missing environment requirements, through outdated libraries, which have dropped backward compatibility,
and finishing on overfitted models, which are good only within specific circumstances and are not applicable in ordinary indoor applications.
This thesis aims to contribute to the field of agent localization and plane segmentation
by reviewing the current development of plane segmentation and measuring its potential on data
that is completely different from the training or testing one.
Additionally, some of the latest models are difficult to get started.
In this work, I also try to ease the workload required to get these projects going for future contributors or researchers.

\section{Content of this work}

The thesis is organized into chapters as follows:

\begin{itemize}
\item \textbf{Chapter 2: Plane Segmentation Development} \ref{chap:plane-segmentation-development}

It reviews different approaches to plane segmentation, including first attempts and new developments.

\item \textbf{Chapter 3: Thesis aims} \ref{chap:thesis-aims}

It lists the objectives set for this work.

\item \textbf{Chapter 4: Implementation} \ref{chap:implementation}

It describes the initiation and operation of selected projects,
including encountered problems and implemented solutions.

\item \textbf{Chapter 5: Results Evaluation} \ref{chap:results-evaluation}

It presents and compares the results obtained from selected models
and discusses their usefulness in everyday situations.

\item \textbf{Chapter 6: Conclusions} \ref{chap:conclusions}

It summarises the results obtained in this work and shows potential directions for future research.

\end{itemize}

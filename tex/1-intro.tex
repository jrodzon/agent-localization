
\chapter{Preface}

We live in the era of automation.
Everywhere robots are taking over the positions, which require not only repetitive actions but also more complicated tasks.
In this environment, there are more and more machines that are operating within some contained spaces like commercial buildings or warehouses.
There is also a growing market for indoor drone usage.
Because of that, there is a need for a means of autonomous localization of these machines, which we will call generically agents in this work.
Recent advances in computer vision and artificial intelligence come in handy and provide some models for localization based on planar segments.

\section{Motivation}

Unfortunately, when one tries to use or apply the latest state-of-the-art models and algorithms there are a lot of obstacles to it—starting from
missing environment requirements, through outdated libraries, which have dropped backward compatibility,
and finishing on overfitted models, which are good only within specific circumstances and are not applicable in ordinary indoor applications.
This thesis aims to contribute to the field of agent localization and plane segmentation
by reviewing the current development of plane segmentation and measuring its potential on data
that is completely different from the training or testing one.
Additionally, some of the latest models are difficult to get started.
In this work, I also try to ease the workload required to get these projects going for future contributors or researchers.

\section{Content of this work}

The content of the chapters in this thesis is organized as follows:

\begin{enumerate}
\item \textbf{Chapter 2: Plane Segmentation Development} \ref{chap:plane-segmentation-development}

This chapter goes through the recent development in plane segmentation.
It compares different approaches and techniques to tackle this problem bringing up relevant literature.

\item \textbf{Chapter 3: Problem Description} \ref{chap:problem-description}

This chapter explains the problem of agent localization and the difficulties in applying it in everyday use.
It also covers the obstacles of running the latest state-of-the-art projects to deepen the research in this matter.

\item \textbf{Chapter 4: Implementation} \label{chap:implementation}

This chapter shows the actions taken to get the related projects up and running and to evaluate their usefulness in indoor applications,
including data preparation and processing.

\item \textbf{Chapter 5: Results Evaluation} \label{chap:results-evaluation}

This chapter presents and compares the results of the selected models.
It also shows how to get the related projects going after the contribution.

\item \textbf{Chapter 6: Conclusions} \ref{chap:conclusions}

This chapter concludes the results and shows potential directions for future work.

\end{enumerate}


\chapter{Project documentation} \label{chap:docs}

\newcommand{\uml}[2] {
	\begin{figure}[H]
		\centering
		\caption{#1}
		\begin{mpost}[mpsettings=input metauml;]
			#2
		\end{mpost}
	\end{figure}
}

\bt

\section{Some clever stuff} \label{sec:cuda}

\bt

\section{API components overview} \label{sec:api-overview}

Nice UML no. 1...

\uml{IsogeometricFEM class} {
	Class.C("IsogeometricFEM") ()
		("+ apply(pde: PDE, bcs: BoundaryConditions,
			base: BsplineBase, solver: Solver, femConfig: FEMConfig)");
	drawObject(C);
}

Nice UML no. 2...

\uml{HAdaptiveIsogeometricFEM interface} {
	Interface.I("HAdaptiveIsogeometricFEM")
		("+ apply(pde: PDE, bcs: BoundaryConditions,
			adaptationThreshold: Double, solver: Solver, femConfig: FEMConfig)");
	classStereotypes.I("<<interface>>");
	drawObject(I);
}


\section{Detailed API specification and class diagrams} \label{sec:api-detail}

And another large UML...

\uml{The overall class diagram} {
	Class.IsogeometricFEM("IsogeometricFEM")
		() ("+ apply(...)");

	Class.PDE("PDE")
		("+ lhsCoefs: Function[1..*]", "+ rhs: Function") ();
	IsogeometricFEM.n = PDE.s + (60, -60);

	Class.BoundaryConditions("BoundaryConditions")
		("+ left: Double", "+ right: Double",
		 "+ leftType: BoundaryType", "+ rightType: BoundaryType") ();
	IsogeometricFEM.s = BoundaryConditions.n + (60, 60);

	Interface.Base("Base")
		("+ count(): Int",
		 "+ eval(...): Double",
		 "+ getSupport(no: Int): Range");
	classStereotypes.Base("<<interface>>");
	IsogeometricFEM.w = BsplineBase.e + (-150, 0);

	AbstractClass.GriddedBase("GriddedBase") ()
		("+ getGrid(): PointSequence",
		 "+ getGridSpan(index: Int): IntRange",
		 "+ getSupport(no: Int): Range");
	Base.n = GriddedBase.s + (-80, -60);

	Class.BsplineBase("BsplineBase") ()
		("+ getGrid(): PointSequence",
		 "+ getGridSpan(index: Int): IntRange",
		 "+ getSupport(no: Int): Range");
	Base.n = BsplineBase.s + (80, -60);

	Interface.Solver("Solver")
		("+ apply(...)");
	classStereotypes.Solver("<<interface>>");
	IsogeometricFEM.s = Solver.n + (-60, 60);

	Class.GaussSolver("GaussSolver")
		() ("+ apply(...)");
	Solver.e = GaussSolver.w + (-60, 0);

	Class.MultifrontalSolver("MultifrontalSolver")
		() ("+ apply(...)");
	Solver.s = MultifrontalSolver.n + (0, 60);


	drawObjects(
		IsogeometricFEM,
		PDE, BoundaryConditions,
		Base, GriddedBase, BsplineBase,
		Solver, GaussSolver, MultifrontalSolver
	);

	link(dependency)(IsogeometricFEM.n -- PDE.s);
	link(dependency)(IsogeometricFEM.n -- BsplineBase.s);
	link(dependency)(IsogeometricFEM.s -- BoundaryConditions.n);
	link(dependency)(IsogeometricFEM.s -- Solver.n);
	link(realization)(GriddedBase.s -- Base.n);
	link(realization)(BsplineBase.s -- Base.n);
	link(realization)(GaussSolver.w -- Solver.e);
	link(realization)(MultifrontalSolver.n -- Solver.s);
}

